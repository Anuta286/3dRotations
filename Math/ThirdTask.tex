%! Author = Админ
%! Date = 26.03.2022

% Preamble
\documentclass[11pt]{article}

% Packages
\usepackage{amsmath}
\usepackage{graphicx}
\usepackage{gensymb}
\usepackage{mathtools}
\graphicspath{{pictures/}}

% Document
\begin{document}
    \section{Problem}
    a) Q-? \newline
    \begin{gather*}
        (EP=a)\\
        \overline{PE}(2-x_0;-y_0;-z_0)\\
        a_x(t)=(2-x_0)t+x_0\\
        a_y(t)=-y_0t+y_0\\
        a_z(t)=-z_0t+z_0\\
    \end{gather*}
    Plane (yz): x=0 \newline
    \begin{gather*}
        0=(2-x_0)t+x_0\\
        t=-\frac{x_0}{2-x_0}\\
        a_x=0\\
        a_y=-y_0(-\frac{x_0}{2-x_0})+y_0=\frac{x_0y_0}{2-x_0}+y_0\\
        a_z=\frac{x_0z_0}{2-x_0}+z_0\\
    \end{gather*}
    Answer: \[Q (0; \frac{x_0y_0}{2-x_0}+y_0; \frac{x_0z_0}{2-x_0}+z_0)\]

    b) It can look like line segment (screen projection) or single point (if it locates on "EP" line).

    c) \[P_0(-1;-3;1), P_1(-2;4;6)\]
    \[\overline{P_0E}(3;3;-1), \overline{P_1E}(4;-4;-6)\]
    \[Q_0(0;-2;\frac{2}{3})\]
    \[Q_1(0;2;3)\]
    Answer: line segment \(Q_0 Q_1\) is drawn on the screen.

    e) \[\overline{ED}(-1;0;1), \overline{EG}(-1;1;1)\]
    Plane \(\alpha\):
    \[\overline{N}=\overline{ED}\times\overline{EG}= \begin{bmatrix}
                                                     -1 \\
                                                      0 \\
                                                     -1
                                                     \end{bmatrix} \]
    \[-x-z=d\]
    E: \[-2+0-0=-2\]
    D: \[-1+0-1=-2\]
    \[d=-2\]
    Equation of plane \(\alpha: x+z=2\)
    \[\overline{P_0P_1}(-1;7;5)\]
    \[x(t)=-t-1\]
    \[y(t)=7t+3\]
    \[z(t)=5t+1\]
    \[P_n(-t-1; 7t+3; 5t+1)\]
    \[x+z=2 => -t-1+5t+1=4t>2 => t>0.5\]
    Answer: So, that portion of the trajectory is hidden, which has \(t>0.5\) in the parametric equation of a line of trajectory.


\end{document}